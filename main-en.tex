\documentclass[conference]{IEEEtran}
\IEEEoverridecommandlockouts

\usepackage[utf8]{inputenc}
\usepackage{cite}
\usepackage{amsmath,amssymb,amsfonts}
\usepackage{algorithmic}
\usepackage{graphicx}
\usepackage{textcomp}
\usepackage{xcolor}
\usepackage{bytefield}
\usepackage{url}

\def\BibTeX{{\rm B\kern-.05em{\sc i\kern-.025em b}\kern-.08em
    T\kern-.1667em\lower.7ex\hbox{E}\kern-.125emX}}
\begin{document}

\title{Enabling Adaptive Source Routing with Custom SRv6 Function}

\author{\IEEEauthorblockN{Yuzuki Ishiyama (ID:2531008)}
  \IEEEauthorblockA{\textit{Yamaki Laboratory} - \textit{Information and Network Engineering}}
}

\maketitle

\begin{abstract}
  Segment Routing over IPv6 (SRv6) has garnered attention as a flexible routing mechanism to meet the diverse requirements of modern networks.
  However, SRv6 has inherent limitations. Routing decisions are made statically at the source, making it impossible to adapt to network conditions that change after packet transmission.
  This paper proposes a novel approach to achieve adaptive source routing using SRv6 Functions.
  By directly encoding conditional logic into the segment list, this method enables dynamic path selection based on real-time network metrics without relying on additional control packets.
  We implemented this approach on Linux routers using extended Berkeley Packet Filter (eBPF) and demonstrated its feasibility through experimental evaluation.
  Compared to a conventional probe-based approach, our method demonstrated 43.1\% higher throughput and more stable network performance.
  The implementation overhead was minimal, with only a 3.2\% increase in protocol size and a 1.04\% throughput reduction due to eBPF execution.
\end{abstract}

\begin{IEEEkeywords}
  service function chaining, segment routing, SRv6, QoS
\end{IEEEkeywords}

\section{Introduction}

Internet Service Providers (ISPs) are increasingly adopting source routing technologies to implement diverse routing policies for various applications \cite{cisco_rakuten_srv6}\cite{softbank_srv6}.
Source routing, where the source specifies the path for packet forwarding, offers improved scalability and controllability compared to traditional hop-by-hop forwarding.
% For example, using source routing, ISPs can efficiently provide high-bandwidth paths for specific applications and low-latency paths for others.
Segment Routing over IPv6 (SRv6) is a protocol that has recently gained attention, implementing source routing by embedding path information in IPv6 extension headers.
Compared to other technologies, SRv6 is widely adopted due to its ease of deployment in IPv6 networks.

However, source routing has a fundamental limitation: paths must be determined at the time of packet transmission, making it unable to adapt to network conditions that change post-transmission.
This paper proposes a method to overcome this limitation by directly encoding conditional logic into the SRv6 segment list.
By leveraging SRv6 Functions, this method enables dynamic path selection based on real-time network metrics without requiring additional control messages.

\section{Background}

\subsection{Traditional IP Routing}

Traditional IP routing faces challenges in implementing complex routing policies. This is because it employs a hop-by-hop forwarding mechanism where each router independently determines the next hop based solely on the destination address in the packet header and its own routing table. Consequently, intermediate routers cannot perform path selection considering application requirements or end-to-end path characteristics.

\subsection{Source Routing and Potential Issues}

Source routing allows the sender to specify the complete path that a packet should traverse through the network.
This paradigm provides application-specific routing with scalability and flexibility. SRv6 implements source routing by adding a Segment Routing Header (SRH) to IPv6 packets \cite{rfc8754}\cite{rfc9256}.
The SRH contains a segment list, which is a sequence of Segment Identifiers (SIDs) representing the nodes to be traversed.
In most cases, SIDs are IPv6 addresses.

However, source routing has a fundamental limitation. Routing decisions are made statically at the source, making it impossible to adapt to network conditions that change after packet transmission.

\subsection{SRv6 Function}

The segment list can also include function calls. These functions, called SRv6 Functions, are instructions that specify operations to be performed when a packet reaches a specific node.
Each instruction is encoded in the format of an IPv6 address and consists of three parts: Locator, Function, and Arguments.
% Figure \ref{fig:example-of-dividing-an-ipv6-address} shows the structure of an IPv6 address.
The Locator points to the node where the Function is executed, and the Function and Arguments specify the operation to be performed at that node.

% show locator, function, and arguments in a bytefield
% | 128 bits                       |
% | 64 bits | 16 bits  | 48 bits   |
% | Locator | Function | Arguments |

% \begin{figure}[htbp]
%   \centering
%   \begin{bytefield}[bitwidth=0.18em]{128}
%     \bitheader{0, 16, 32, 48, 64, 80, 96, 112, 128} \\
%     \bitbox{64}{\small LOC} & \bitbox{16}{\small FUNC} & \bitbox{48}{\small ARG} \\
%   \end{bytefield}
%   \caption{Example of IPv6 Address Division}
%   \label{fig:example-of-dividing-an-ipv6-address}
% \end{figure}

\section{Proposed Method}
\subsection{Encoding Conditions in SRv6 Function}

The basic concept is to include multiple potential paths in the segment list, along with logic to decide which path to take based on network metrics.

To implement conditional paths in the segment list, we define two SRv6 Functions: \textit{skip\_if} and \textit{skip}.
\textit{skip\_if} conditionally skips a specified number of segments based on a metric condition, such as bandwidth.
\textit{skip} simply skips a specified number of segments.
For example, if the bandwidth between router A and B exceeds a threshold, forward to router C, otherwise forward to router B, the segment list would be \{ A:\textit{skip\_if}(1, condition), B:\textit{skip}(1), C \}.

% [Below text] Priority: Low
% To reduce noise in metric measurement, metrics are smoothed using an Exponential Moving Average (EMA).
% The EMA is calculated by the following formula:
% \begin{equation}
%   \text{EMA}_t = \alpha \cdot x_t + (1 - \alpha) \cdot \text{EMA}_{t-1}
% \end{equation}
% where \(x_t\) is the current measurement, \(\text{EMA}_{t-1}\) is the previous EMA value, and \(\alpha\) is the smoothing factor.

\section{Evaluation}
\subsection{Experimental Setup}

\begin{figure}[t]
  \centering
  \includegraphics[width=0.7\linewidth]{./figures/topo.pdf}
  \caption{Network Topology}
  \label{fig:network-topology}
\end{figure}

% == OLD START ==
% We evaluated our method using a network topology with multiple potential paths between source and destination nodes. The Linux routers ran our eBPF implementation, and network namespaces, one of the linux features, are used for virtualization, tc for bandwidth/delay control, and iperf3 for traffic generation.
% The proposed method was evaluated using two scenarios:
% \begin{itemize}
%   \item \textbf{Scenario 1}: A single TCP flow that exceeds the threshold, requiring path switching.
%   \item \textbf{Scenario 2}: Two TCP flows competing for bandwidth, with dynamic path selection.
% \end{itemize}
% For comparison, the probe-based approach was also implemented using a separate control packet to probe the network condition.
% == OLD END ==

We evaluated our proposed method in terms of network quality and overhead.

The evaluation of network quality was conducted using two scenarios with the network topology shown in Figure \ref{fig:network-topology}.
Each circle represents a router. A delay of 2ms was set between router 1-2, and a bandwidth limit of 1 Gbps was set between router 2-3 and router 3-4.
In the first scenario, a TCP flow is sent from router 1 to router 6. The segment list is configured such that if the bandwidth between routers exceeds 900 Mbps, the excess traffic is forwarded to router 3-4.
In the second scenario, the TCP flow between router 1-6 and the branching condition are the same, but an additional TCP flow is sent from router 7 to router 8.
For comparison, a conventional probe-based method was also implemented. This is a typical approach that uses separate control packets to investigate network conditions.

The overhead evaluation was performed by comparing TCP throughput with and without SRv6 Functions in the segment list.

\subsection{Results}

% == OLD START ==
% In Scenario 1, the proposed method achieved 43.1\% higher throughput compared to the conventional probe-based approach.
% The performance improvement was attributed to the immediate path switching based on real-time metrics, which avoids TCP congestion control.
% In Scenario 2, our method maintained more stable bandwidth utilization across flows compared to the probe-based approach.
% The implementation overhead was minimal:
% \begin{itemize}
%   \item Protocol size increase: 3.2\%
%   \item Throughput reduction due to eBPF execution: 1.04\%
% \end{itemize}
% == OLD END ==

The evaluation results regarding network quality are presented. Figure \ref{fig:scenario-1} shows the throughput of the conventional method and the proposed method in Scenario 1. The proposed method achieved 43.1\% higher throughput compared to the conventional probe-based approach.
Furthermore, Figure \ref{fig:scenario-2} shows the throughput in Scenario 2. The proposed method achieved more stable throughput compared to the conventional probe-based approach.

Next, the evaluation results regarding overhead are presented. TCP throughput showed a 1.04\% decrease when SRv6 Functions were present in the segment list compared to when they were not.

\begin{figure}[t]
  \centering
  \includegraphics[width=0.7\linewidth]{./figures/scenario-1.pdf}
  \caption{Scenario 1: TCP Throughput}
  \label{fig:scenario-1}
\end{figure}

\begin{figure}[t]
  \centering
  \includegraphics[width=0.7\linewidth]{./figures/scenario-2.pdf}
  \caption{Scenario 2: TCP Throughput}
  \label{fig:scenario-2}
\end{figure}

% == OLD START ==
% We proposed and implemented an adaptive source routing method using SRv6 Functions that enables dynamic path selection based on real-time network metrics.
% By encoding conditional logic directly into the Segment List, each router can make forwarding decisions based on the current network conditions.
% Experimental results demonstrated higher throughput and more stable performance compared to a conventional probe-based approach with minimal overhead.
% Future work includes supporting additional metrics and more complex conditions.
% == OLD END ==

\section{Discussion}

The proposed method presents a novel approach to achieve adaptive source routing by leveraging SRv6 Functions and directly embedding conditional logic within the segment list.
The results of the evaluation experiments showed that this method demonstrated significant improvements in both network throughput and stability compared to conventional probe-based methods.
In particular, the 43.1\% throughput improvement observed in Scenario 1 is attributed to the proposed method's ability to detect changes in network conditions in real-time and immediately optimize paths. This makes it possible to avoid delays in state detection and the resulting inefficient operation of TCP congestion control mechanisms, which are common in probe-based methods.
Similarly, the improvement in throughput stability in Scenario 2 is also considered to be due to this rapid adaptability.
In terms of implementation, the execution overhead of this method using eBPF was very small, with a throughput decrease of 1.04\%, and the increase in protocol header size was also limited. These results suggest that the proposed method has sufficient practicality even in real-world environments.
While this study primarily used bandwidth as the decision criterion, the realization of more complex conditional logic combining diverse network metrics such as delay, jitter, and packet loss rate is anticipated in the future.

\section{Conclusion}

This paper proposed a novel approach to achieve adaptive source routing using SRv6 Functions.
By directly encoding conditional logic into the segment list, this method enables dynamic path selection based on real-time network metrics without relying on additional control packets.
We implemented this approach on Linux routers using extended Berkeley Packet Filter (eBPF) and demonstrated its feasibility through experimental evaluation.
Compared to a conventional probe-based approach, our method demonstrated higher throughput and more stable network performance.
The implementation overhead was minimal, and the increase in protocol size was slight.
Future work includes supporting additional metrics and more complex conditions.

\bibliographystyle{IEEEtran}
\bibliography{refs}

\end{document}
